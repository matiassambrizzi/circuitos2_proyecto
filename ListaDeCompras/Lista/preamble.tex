\usepackage[spanish]{babel}
\usepackage[utf8]{inputenc}
\usepackage[T1]{fontenc}
\usepackage{amsmath,amssymb,amsthm,mathrsfs,amsfonts,dsfont}
\usepackage{graphicx}
\usepackage{color}
\usepackage{multirow}
\usepackage{sidecap}
\usepackage{float}
\usepackage{cleveref} 
\usepackage{parskip} 
\usepackage[colorinlistoftodos]{todonotes} %Para comentarios con \todo
\usepackage{ gensymb }
%Paquetes Matematicos
\usepackage{amsmath,amsfonts,amssymb}
\usepackage[justification=centering]{caption}
%Dimensiones
%\usepackage[left=3cm,right=3cm,bottom=3.5cm,top=3.5cm]{geometry}
\usepackage{fancyhdr}
\pagestyle{fancy}
\fancyhf{}
\rhead{\it }
\lhead{\it }
\cfoot{Página \thepage }
\usepackage{etoolbox}
\makeatletter
\newenvironment{absolutelynopagebreak}
  {\par\nobreak\vfil\penalty0\vfilneg
   \vtop\bgroup}
  {\par\xdef\tpd{\the\prevdepth}\egroup
   \prevdepth=\tpd}
\usepackage{lipsum} % for dummy text only
\usepackage{caption}
\usepackage{subcaption}
\captionsetup[table]{name=Tabla}

\usepackage[a4paper,top=3cm,bottom=2cm,left=3cm,right=3cm,marginparwidth=1.75cm]{geometry}%% Sets page size and margins
\usepackage{wrapfig}
\usepackage{amsmath}

\numberwithin{equation}{section}   %EC 1.1, 1.2, 2.1, 2.2, etc... auomatico por seccion
\numberwithin{table}{section}   %EC 1.1, 1.2, 2.1, 2.2, etc... auomatico por seccion
\numberwithin{figure}{section}  

\usepackage{array}
\linespread{1.15}  %Para interlineado%
\usepackage{physics}
 %Este paquete tiene derivadas mas chetas:
%	\dv{Q}{t} = \dv{s}{t}		 	para derivadas comun
%	\dv[n]{Q}{t} = \dv[n]{s}{t} 	para derivadas comun sucesivas (segunda derivada, tercera, etc..)
%	
%	\pdv{Q}{t} = \pdv{s}{t}  		lo mismo para derivadas parciales
%	\pdv[n]{Q}{t} = \pdv[n]{s}{t} 
%	\pdv{Q}{x}{t} = \pdv{s}{x}{t} 	para derivadas parciales multiples distintas
\usepackage{listings} %%Para copiar y pegar codigo
%%begin{lstlisting}....end{lstlisting}
%%\lstinputlisting{nombre_fichero_codigo.py}
